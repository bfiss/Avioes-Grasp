\documentclass[a4paper,10pt]{abnt} %
\usepackage[utf8x]{inputenc}
\usepackage[brazil]{babel}
\usepackage{graphicx}
\usepackage{listings}
\usepackage[usenames,dvipsnames]{color}
\usepackage{verbatim, amssymb, latexsym, amsmath, mathrsfs}
\usepackage{url}
\usepackage{subfigure}
\usepackage{multicol}
\usepackage{framed}
\usepackage{array}
\usepackage{float}
\usepackage[section] {placeins}
\usepackage[table]{xcolor}
\usepackage{tipa}
\usepackage{listings}

\newcommand{\minimize}{
  \iflanguage{portuges}{minimiza}{}%
  \iflanguage{english}{minimize}{}%
  \iflanguage{german}{minimiere}{}%
}
\newcommand{\maximize}{
  \iflanguage{portuges}{maximiza}{}%
  \iflanguage{english}{maximize}{}%
  \iflanguage{german}{maximiere}{}%
}
\newcommand{\mexists}{
  \iflanguage{portuges}{existe}{}%
  \iflanguage{english}{exists}{}%
  \iflanguage{german}{existiert}{}%
}
\newcommand{\subjectto}{
  \iflanguage{portuges}{sujeito a}{}%
  \iflanguage{english}{subject to}{}%
  \iflanguage{german}{so dass}{}%
}
\makeatletter
\newcommand{\minproblem}{\@ifstar\minproblemstar\minproblemplain}
\newcommand{\minproblemplain}[2]{
  \begin{align}
    \textbf{\minimize}\qquad & #1\\
    \textbf{\subjectto}\qquad & #2
  \end{align}
}
\newcommand{\minproblemstar}[2]{
  \begin{align*}
    \textbf{\minimize}\qquad & #1\\
    \textbf{\subjectto}\qquad & #2
  \end{align*}
}
\newcommand{\maxproblem}{\@ifstar\maxproblemstar\maxproblemplain}
\newcommand{\maxproblemplain}[2]{
  \begin{align}
    \textbf{\maximize}\qquad & #1\\
    \textbf{\subjectto}\qquad & #2
  \end{align}
}
\newcommand{\maxproblemstar}[2]{
  \begin{align*}
    \textbf{\maximize}\qquad & #1\\
    \textbf{\subjectto}\qquad & #2
  \end{align*}
}
\newcommand{\existsproblem}{\@ifstar\existsproblemstar\existsproblemplain}
\newcommand{\existsproblemplain}[2]{
  \begin{align}
    \textbf{\mexists}\qquad & #1\\
    \textbf{\subjectto}\qquad & #2
  \end{align}
}
\newcommand{\existsproblemstar}[2]{
  \begin{align*}
    \textbf{\mexists}\qquad & #1\\
    \textbf{\subjectto}\qquad & #2
  \end{align*}
}
\makeatother

\lstset{tabsize=2}
\newcommand{\bl}{\begin{lstlisting}}
%\newcommand{\el}{\end{lstlisting}}

\autor{Bruno Coswig Fiss \\ Kauê Soares da Silveira} %\and
\instituicao{Universidade Federal do Rio Grande do Sul}
\orientador[Professor:]{Marcus Rolf Peter Ritt}
\titulo{GRASP aplicado ao problema de aterrissagem de aviões}
\comentario{INF05010 – Otimização Combinatória}
\data{1º de julho de 2010}

\begin{document}

%\maketitle
\folhaderosto
\sumario
%\listadetabelas
%\listadefiguras

%\begin{abstract}
%\begin{resumo}
%\end{abstract}
%\end{resumo}

\chapter{Descrição do problema}

\section{Descrição formal}

O problema de aterrissagem de aviões consiste em definir um momento no tempo para a aterrissagem de cada avião $ i \in P $, sendo $P$ o conjunto de aviões. Cada avião possui os seguintes dados:

$ E_i$: Momento mais prematuro em que o avião i pode realizar pouso.

$T_i$: Momento ideal para pouso do avião i.

$L_i$: Momento mais tardio em que o avião i pode realizar pouso.

$g_i$: Penalidade por unidade de tempo da diferença do pouso para o tempo ideal se o avião chegar mais cedo do que o ideal.

$h_i$: Penalidade por unidade de tempo da diferença do pouso para o tempo ideal se o avião chegar mais tarde do que o ideal.

$S_{ij}$: Distância de tempo requerida após o pouso do avião i para que o avião j possa pousar.

O objetivo é encontrar uma solução com somatório de todas as penalidades mais baixo possivel.

\section{Formulação como um programa inteiro}

O seguinte programa inteiro descreve o problema acima descrito:

\minproblemstar{\sum_{i \in P } g_i  \alpha_i + h_i \beta_i}
                         {   x_i = - \alpha_i + \beta_i + T_i , & \forall i \in P \\
                         &  E_i \le x_i \le L_i, & \forall i \in P \\ 
                         &  x_j - x_i \ge S_{ij}\delta_{ij} + (E_j - L_i)\delta_{ji}, & \forall i,j \in P \\
		 &  \delta_{ij} + \delta_{ji} = 1, & \forall i,j \in P \\
                         &  x_i \ge 0, x_i \in \mathbb{R}, & \forall i \in P \\
                         &  \alpha_i \ge 0, \alpha_i \in \mathbb{R}, & \forall i \in P \\
                         &  \beta_i \ge 0, \beta_i \in \mathbb{R}, & \forall i \in P \\
                         &  \delta_{ij} \in \mathbb{B}, & \forall i,j \in P}

A variável $x_i$ representa o momento da aterrissagem do avião $i$. As variáveis $\alpha_i$ e $\beta_i$ indicam a diferença para menos ou mais, respectivamente, do momento da aterrissagem para o momento ideal de aterrissagem do avião $i$. $\delta_{ij}$ é uma variável binária que indica se o avião $i$ aterrissa antes do avião $j$.

A formulação do programa inteiro foi realizada pelos autores antes da leitura de artigos relacionados com o problema, que por sua vez continham uma formulação similar. Para facilitar a compreensão, tornamos a simbologia utilizada por nós semelhante à utilizada nesses artigos \cite{768657,din}.

\section{Representação de um solução}

O programa inteiro que descreve o problema aqui tratado é misto, possuindo variáveis reais além de inteiras. Como o objetivo desse trabalho é aplicar uma meta-heurística a um problema inteiro, separamos a solução em duas partes, uma inteira e outra real.

A parte inteira da solução é a que contém as variáveis $\delta_{ij}$, que descrevem, em conjunto, a ordem de chegada dos aviões. O caminho inverso também é válido, ou seja, uma determinada ordem de chegada dos aviões descreve completamente as variáveis $\delta_{ij}$. Já a parte linear contém as demais variáveis.

Portanto, representaremos uma solução como uma sequência de aviões que descreve a ordem de chegada desses. Após obter a solução inteira do problema, basta resolver o programa linear restante para obter a solução completa do problema.

\chapter{Algoritmo proposto}

\section{Idéia geral}

Nosso algoritmo utiliza a biblioteca GLPK (GNU linear programming kit) para resolução das partes lineares do problema sendo tratado.

Primeiramente, definimos o problema inteiro utilizando os dados lidos da entrada e as funções para criação de restrições disponibilizadas pela biblioteca GLPK. Após esse passo é possível utilizar o \emph{solver} interno do GLPK para resolver o problema inteiro, utilizando o método branch-and-cut. Essa característica foi requisitada na especificação do trabalho da disciplina, e também permite a comparação entre as soluções atingidas pelo algoritmo do GLPK e pelo método GRASP, além da comparação entre o desempenho na obtenção dessas.

Após isso, utilizamos a meta-heurística GRASP para criar soluções para a parte inteira do problema, ou seja, sequências de aviões que definem a ordem de chegada desses. Para cada solução inteira, modificamos as restrições do problema inteiro criado no primeiro passo de forma que ele respeite a ordem presente na sequência e torne-se, por consequência, linear. Cada programa linear gerado é então resolvido através do método simplex utilizando uma rotina da biblioteca GLPK.

Isso é feito para cada solução inteira gerada pelo método GRASP, tanto para as soluções geradas durante a construção gulosa aleatória quanto para as soluções geradas na busca local (soluções vizinhas da solução atual). A melhor solução encontrada é armazenada e impressa ao fim do algoritmo.

\section{GRASP}

Decidimos utilizar esse método pois ele cria soluções, de forma aleatória, utilizando recursos e conhecimento sobre o problema, para, por exemplo, construir um custo heurístico da adição de um elemento a uma solução parcial e assim construir soluções mais efetivamente. Isso inicialmente pareceu ser eficiente e inteligente, e nossas conclusões podem ser vistas na seção \ref{Analise}.

\subsection{Criação de soluções}

A utilização de uma meta-heurística como o GRASP tem como primeira etapa a definição dos conjuntos, funções e parâmetros necessários para a definição do problema combinatório a ser resolvido. Utilizaremos os conjuntos, funções e parâmetros citados no artigo de Resende e Ribeiro \cite{Resende02greedyrandomized} para formalização do nosso método.

Seguindo o artigo, os conjuntos $E$, $F \subseteq 2^{E}$ e a função $f : 2^{E} \rightarrow \mathbb{R}$ foram definidos. O conjunto $E$ de elementos que possivelmente fazem parte da solução é o conjunto de variáveis $\delta_{ij}$. O conjunto $F$ é o conjunto de todos os conjuntos de variáveis $\delta_{ij}$ tal que, caso as variáveis do conjunto sejam todas definidas como tendo valor 1, e as variáveis $\delta_{ij}$ que não estiverem no conjunto forem definidas como tendo valor 0, existe uma solução para o programa linear resultante. A função de custo $f$ mapeia um dado conjunto de variáveis $\delta_{ij}$ para o custo da solução mínima do programa linear em que as variáveis pertencentes ao conjunto são definidas como 1 e as demais como 0.

Além disso, as meta-construções do algoritmo GRASP precisam ser instanciadas, e essas são a representação da solução e de cada elemento a ser adicionado a essa, a RCL (Restricted Candidate List), e o parâmetro $\alpha$, além dos parâmetros para controle do término do algoritmo, que aqui são $iter_{max}$ e $time_{max}$.

A representação de uma solução será a ordem da chegada dos aviões, que por sua vez será representada por uma sequência de aviões, na qual um avião chega antes de outro se estiver mais à esquerda na sequência. Os elementos a serem adicionados, passo a passo, para a construção da solução, são os próprios aviões. Dado um parâmetro $\alpha$, em um determinado passo da construção da solução, nossa RCL é o conjunto de aviões que estão posicionados,  na sequência ideal, a até $\alpha$ posições da posição a ser adicionada na solução atual e que, se forem adicionados, não tornarão a solução inviável. E.g., dada a sequência ideal 1,2,6,4,5,3, o parâmetro $\alpha = 2$ e a solução atual ser 1,4, a RCL seria $\{6,2,5\}$ desde que a adição de qualquer um desses elementos à solução atual mantivesse-a viável.

A sequência ideal é definida inicialmente como a ordem de chegada dos aviões caso todos eles chegassem no seu tempo ideal. Se uma dada solução obtiver um custo menor do que a sequência ideal, a sequência que define a nova solução se tornará a sequência ideal.

Para efetivamente testar a sequência ideal, nossa primeira iteração utiliza $\alpha = 0$, ou seja, uma construção totalmente gulosa.

Após a realização de testes e do estudo da variação Reactive GRASP da meta-heurística GRASP, decidimos implementar parte das variações no nosso algoritmo. A variação implementada é simples: o parâmetro $\alpha$ é, a cada iteração, sorteado aleatoriamente entre 0 e um novo parâmetro, chamado $\alpha_{max}$.

Os parâmetros $iter_{max}$ e $time_{max}$ definem, respectivamente, o número máximo de iterações da e o tempo máximo em segundos gastos na heurística GRASP.

\subsection{Vizinhança e busca local}

A busca local pode ser feita de duas formas: com o primeiro incremento ou com o melhor incremento. Decidimos utilizar o primeiro incremento, ou seja, pesquisamos a vizinhança de uma dada solução em busca de uma solução melhor e, quando encontramos a primeira, passamos a considerar essa nova solução encontrada como a solução atual, buscando agora uma solução melhor na vizinhança da atual.

Resta definirmos o que é a vizinhança de uma solução, o que é feito a seguir: uma dada solução, representada por uma sequência de aviões na qual dois aviões adjacentes foram trocados, um sendo colocado no lugar do outro, e nenhuma outra modificação ocorrendo, é vizinha da solução original, sem a troca dos aviões. A vizinhança de uma dada solução é simplesmente o conjunto de todos os vizinhos dessa.

\chapter{Experimentos}

\section{Configurações}

Os testes realizados tiveram por objetivo testar a eficácia da nossa abordagem dadas diferentes sementes aleatórias e parâmetros $\alpha_{max}$. Testamos, é claro, esses parâmetros para cada caso de entrada.

Para repetir os testes, basta compilar o arquivo avioes.c, ligá-lo à biblioteca GLPK criando um executável chamado Avioes.exe, compilar o arquivo experimentos.c, e executar o arquivo resultante da ligação e montagem do objeto proveniente da compilação de experimentos.c. As compilações podem ser realizadas com o compilador gcc. A ligação e montagem dependem do sistema operacional utilizado, além da localização da biblioteca GLPK.

Os resultados serão concatenados ao conteúdo do arquivo out.txt no formato de uma tabela no estilo \LaTeX com o mesmo formato da tabela \ref{tab}, porém sem o desvio para a solução ótima.

\section{Resultados}

Em todos os testes o limite de iterações do método foi 100 e o limite de tempo foi de 20 segundos. Vários casos passaram de 20 segundos pois o limite de tempo só é verificado ao final de cada iteração, que pode ser longa.

\begin{table}

\centering

\caption{Tabela de resultados dos experimentos}

\begin{tabular}{|c|c|c|c|c|c|}

\hline % este comando coloca uma linha na tabela

Caso & $\alpha_{max}$ & Semente & Solução & Tempo utilizado (s) & Desvio para solução ótima (\%)  \\

\hline
airland1 & 1 & 14766 & 700 & 1 & 0 \\ 
airland1 & 2 & 20024 & 700 & 2 & 0 \\ 
airland1 & 3 & 19673 & 700 & 1 & 0 \\ 
airland1 & 4 & 23915 & 700 & 2 & 0 \\ 
airland1 & 5 & 710 & 700 & 2 & 0 \\ 
\hline
airland2 & 1 & 22574 & 1.480 & 5 & 0 \\ 
airland2 & 2 & 26494 & 1.480 & 9 & 0 \\ 
airland2 & 3 & 11040 & 1.480 & 8 & 0 \\ 
airland2 & 4 & 27382 & 1.480 & 9 & 0 \\ 
airland2 & 5 & 11503 & 1.500 & 10 & 1.33 \\ 
\hline
airland3 & 1 & 15751 & 820 & 18 & 0 \\ 
airland3 & 2 & 14185 & 820 & 21 & 0 \\ 
airland3 & 3 & 31028 & 820 & 21 & 0 \\ 
airland3 & 4 & 20012 & 820 & 21 & 0 \\ 
airland3 & 5 & 17962 & 820 & 21 & 0 \\ 
\hline
airland4 & 1 & 26543 & 2.520 & 18 & 0 \\ 
airland4 & 2 & 23571 & 2.520 & 21 & 0 \\ 
airland4 & 3 & 21435 & 2.520 & 21 & 0 \\ 
airland4 & 4 & 3896 & 2.520 & 21 & 0 \\ 
airland4 & 5 & 21136 & 2.520 & 21 & 0 \\ 
\hline
airland5 & 1 & 20362 & 3.100 & 21 & 0 \\ 
airland5 & 2 & 19016 & 3.100 & 21 & 0 \\ 
airland5 & 3 & 28928 & 3.100 & 21 & 0 \\ 
airland5 & 4 & 13363 & 3.100 & 21 & 0 \\ 
airland5 & 5 & 4087 & 3.100 & 21 & 0 \\ 
\hline
airland6 & 1 & 27207 & 24.442 & 1 & 0 \\ 
airland6 & 2 & 14705 & 24.442 & 1 & 0 \\ 
airland6 & 3 & 31960 & 24.442 & 2 & 0 \\ 
airland6 & 4 & 32355 & 24.442 & 1 & 0 \\ 
airland6 & 5 & 15148 & 24.442 & 1 & 0 \\ 
\hline
airland7 & 1 & 4734 & 1.550 & 12 & 0 \\ 
airland7 & 2 & 11699 & 1.550 & 13 & 0 \\ 
airland7 & 3 & 23048 & 1.550 & 13 & 0 \\ 
airland7 & 4 & 30431 & 1.550 & 14 & 0 \\ 
airland7 & 5 & 10183 & 1.550 & 14 & 0 \\ 
\hline
airland8 & 1 & 30714 & 1.950 & 21 & 0 \\ 
airland8 & 2 & 13072 & 1.950 & 33 & 0 \\ 
airland8 & 3 & 16746 & 1.950 & 40 & 0 \\ 
airland8 & 4 & 20341 & 1.950 & 21 & 0 \\ 
airland8 & 5 & 21053 & 1.950 & 28 & 0 \\ 
\hline

\end{tabular}
\label{tab}
\end{table} 

\section{Análise} \label{Analise}

Como pode-se perceber, a abordagem gulosa é extremamente eficaz na resolução desse problema, sendo que a utilização de $\alpha = 0$ na geração de uma solução gera a melhor solução ou um vizinho próximo dessa em 7 dos 8 casos de entrada testados. No caso airland2, a solução inicial gera, por busca local, uma solução com custo de 1500, porém com algumas iterações do método, utilizando um $\alpha$ variável, a melhor solução, com custo de 1480, é quase sempre encontrada. A única exceção ocorre quando $\alpha_{max}$ possui valor 5 (ou maior), pois, como airland2 considera 15 aviões no total, a criação de soluções fica muito próxima da geração aleatória, o que mostra-se menos efetivo. Com mais iterações (200), entretanto, o resultado volta a ser 1480, visto que, com menor probabilidade de gerar uma solução boa, mais tentativas são necessárias para isso.

Pode-se concluir que o método é bastante afetado pela escolha da função que determina o custo incremental da adição de um dado elemento à solução que está sendo gerada, ou seja, a parte gulosa do algoritmo. No caso desse trabalho, a função, que determina esse custo pela posição do avião na sequência ideal, parece ter sido bem escolhida, o que gerou soluções ótimas em quase todos os casos.

Além disso, o parâmetro $\alpha$ é vital, o que nos levou a decidir não utilizar o método GRASP padrão, mas uma variação do Reactive GRASP, que decide o valor de $\alpha$ de forma aleatória com probabilidade uniforme para todos os $\alpha$ entre 0 e $\alpha_{max}$. Essa escolha também parece ter ajudado no sucesso da abordagem.

\bibliography{relatorio}{}
\bibliographystyle{plain}

\end{document}

